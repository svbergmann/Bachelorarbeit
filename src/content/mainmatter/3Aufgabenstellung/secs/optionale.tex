\section{Optionale Anforderungen}\label{sec:optionale-anforderungen}
	Das \lfidSystem{} stellt Daten bereit,
	welche von verschiedenen Programmen unterschiedlich interpretiert werden können.
	Diese Anbindung an die Datenbank erfolgt momentan
	für jedes einzelne Programm noch über einen direkten Datenbankzugriff
	und wird zudem aktuell noch nicht direkt im Flugsimulator genutzt.
	Die Lösung soll sein,
	alle Programme und Flugsimulatorkomponenten,
	welche diese Daten benötigen,
	so umzustellen,
	dass diese sich bei dem \gls{webservice} \glslink{autorisierung}{autorisieren} müssen,
	um damit auch Daten abfragen zu können.
	Bei den Programmen,
	welche nur Leserechte benötigen,
	kann dann durch die \gls{autorisierung} ein anderes Zugriffslevel zugewiesen werden,
	als bei Programmen,
	welche tatsächlich dazu befugt sind,
	Daten zu ändern.
	Hier würde dann natürlich auch direkt mitspielen,
	dass für jede Entität alle \httpMethoden{} bereitgestellt werden müssen,
	um alle \crudOperationen{} abbilden zu können.

	Zudem ist der \nameref{subsec:autorisierungsserver} bisher nur ein Kommandozeilenprogramm,
	wie auch der \gls{webservice},
	was eventuell auch noch durch eine grafische Oberfläche
	-- vor allem für die Einrichtung der Rechteverteilung --
	unterstützt werden könnte.