\chapter{Ergebnis und Schlussfolgerungen}\label{ch:ergebnis-und-schlussfolgerungen}

	Die Gestaltung der \gls{authentifizierung} war
	-- wie im \nameref{subsec:autorisierungsserver} zu sehen --
	erfolgreich und realisierbar,
	sowie auch die Bereitstellung einiger Methoden zur Abfrage der Daten.
	Grundlegend konnten also die \hyperref[sec:obligatorische-anforderungen]{obligatorischen Anforderungen} erfüllt werden.
	Das Demoprogramm fragt im Hintergrund genau die Daten strukturiert ab,
	welche von einem echten \gls{fms} benötigt werden.
	Die Benutzeroberfläche ist dabei eher zweitrangig geblieben,
	da verschiedene \glspl{fms} mit unterschiedlichen Anzeigen existieren.
	Wichtig hierbei ist nur,
	dass die angefragten Daten kaskadenartig präsentiert werden.

	Laut dem \nameref{fig:timeOfRequests} braucht die webbasierte Anfrage
	mit einer \gls{autorisierung} ca. $5ms\pm2ms$.
	Damit unterscheidet sich die Dauer kaum zur \glslink{autorisierung}{unautorisierten} Anfrage,
	woraus schlussgefolgert werden kann,
	dass sich eine \gls{autorisierung} nicht negativ auf die Antwortzeiten auswirkt.
	Offensichtlich ist die direkte \gls{sql}-Abfrage vor allem nach der ersten Anfrage viel schneller,
	allerdings ist genau dies nicht gewünscht.
	Bis auf den \gls{webservice} soll kein anderes Programm
	direkte \gls{sql}-Statements an die Datenbank absetzen dürfen.
	Wenn also nur noch der \gls{webservice} direkten Zugriff auf die Datenbank hat,
	wird das ganze System durch die Einheitlichkeit
	und der eliminierten Fehlerquelle von mehreren gleichzeitigen Anfragen stabiler
	und auch schneller.