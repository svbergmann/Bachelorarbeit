\subsubsection{Anforderungen}\label{subsubsec:anforderungen-app.3.1}
	Die Anforderungen an \glspl{webservice} sind wiederum in drei Bereiche unterteilt
	und zudem sollten immer Zuständigkeiten existieren,
	sodass eine Stelle immer \enquote{grundsätzlich zuständig} ist
	und optional auch weitere Zuständigkeiten zugeordnet werden können.

	\myparagraph[APP.3.1]{Basis-Anforderungen}
	Vor allem die Basis Anforderungen MÜSSEN
	laut dem \gls{bsi} erfüllt werden\footcite[Vgl.][APP.3.1 S. 3]{holgerschildt2022}.

	\mysubparagraph[APP.3.1.A1]{APP.3.1.A1 Authentisierung}
	\glspl{webservice} MÜSSEN immer so gestaltet sein,
	dass der Zugriff auf gesperrte Ressourcen nur über eine \gls{authentisierung} erfolgen kann.
	Die Methode hierfür SOLLTE protokolliert werden,
	wobei der Betrieb zudem eine Höchstgrenze
	für fehlgeschlagene Anmeldeversuche festsetzen muss\footcite[Vgl.][APP.3.1 S. 3]{holgerschildt2022}.

	\mysubparagraph[APP.3.1.A4]{APP.3.1.A4 Kontrolliertes Einbinden von Dateien und Inhalten}
	Diese Anforderung entfällt für diesen \gls{webservice},
	da keine Daten hochgeladen werden.

	\mysubparagraph[APP.3.1.A7]{APP.3.1.A7 Schutz vor unerlaubter automatisierter Nutzung}
	Es MUSS garantiert sein,
	dass der \gls{webservice} vor \glslink{autorisierung}{unautorisierter},
	automatisierter Nutzung geschützt ist,
	während das Verhalten der Maßnahmen auf \glslink{autorisierung}{autorisierte} Nutzer
	auch berücksichtigt werden MUSS\footcite[Vgl.][APP.3.1 S. 3]{holgerschildt2022}.

	\mysubparagraph[APP.3.1.A14]{APP.3.1.A14 Schutz vertraulicher Daten}
	Vertrauliche Daten,
	wie beispielsweise Zugangsdaten,
	MÜSSEN serverseitig mit Hilfe von \gls{saltedHashVerfahren} abgesichert sein,
	um diese vor \glslink{autorisierung}{unautorisiertem} Zugriff schützen zu können.

	\myparagraph[APP.3.2]{Standard-Anforderungen}
	Standard-Anforderungen SOLLTEN grundsätzlich zusammen
	mit den \nameref{par:APP.3.1} erfüllt sein\footcite[Vgl.][APP.3.2 S. 4]{holgerschildt2022}.

	\mysubparagraph[APP.3.1.A8]{APP.3.1.A8 Systemarchitektur [Beschaffungsstelle]}
	Bereits während der Planung eines \glspl{webservice} SOLLTEN Sicherheitsrichtlinien
	und -konzepte beachtet werden\footcite[Vgl.][APP.3.1 S. 4]{holgerschildt2022}.

	\mysubparagraph[APP.3.1.A9]{APP.3.1.A9 Beschaffung von Webanwendungen und Webservices}
	Als Ergänzung zu den anderen Anforderungen stellt das IT-Grundschutzkompendium eine Liste bereit,
	welche Eigenschaften bei der Beschaffung eines \glspl{webservice} zusätzlich
	beachtet werden SOLLTEN\footcite[Vgl.][APP.3.1 S. 4]{holgerschildt2022}:
	\begin{compactitem}
		\item sichere Eingabevalidierung und Ausgabekodierung
		\item sicheres Session-Management
		\item sichere kryptografische Verfahren
		\item sichere Authentisierungsverfahren
		\item sichere Verfahren zum serverseitigen Speichern von Zugangsdaten
		\item geeignetes Berechtigungsmanagement
		\item ausreichende Protokollierungsmöglichkeiten
		\item regelmäßige Sicherheitsupdates durch den Entwickler der Software
		\item Schutzmechanismen vor verbreiteten Angriffen auf \webApplications{} und \glspl{webservice}
		\item Zugriff auf den Quelltext der \webApplication{} oder des \glspl{webservice}
	\end{compactitem}

	\mysubparagraph[APP.3.1.A11]{APP.3.1.A11 Sichere Anbindung von Hintergrundsystemen}
	Hintergrundsysteme auf denen Funktionen und Daten ausgelagert sind,
	SOLLTEN einzig über definierte Schnittstellen ansprechbar sein.
	Zudem SOLLTE bei netz- und standortübergreifenden Anwendungen die Kommunikation
	\glslink{authentisierung}{authentisiert}
	und verschlüsselt ablaufen\footcite[Vgl.][APP.3.1 S. 4]{holgerschildt2022}.

	\mysubparagraph[APP.3.1.A12]{APP.3.1.A12 Sichere Konfiguration}
	Der Zugriff bzw.\ die Anfragen auf Ressourcen
	und Methoden eines \glspl{webservice} SOLLTEN derart eingeschränkt sein,
	sodass nur über vorher definierte Pfade kommuniziert werden kann.
	Alle anderen Methoden und nicht benötigten Ressourcen SOLLTEN deaktiviert werden.

	\mysubparagraph[APP.3.1.A21]{APP.3.1.A21 Sichere HTTP-Konfiguration bei Webanwendungen}
	Folgende Response-Header SOLLTEN grundsätzlich
	immer gesetzt sein\footcite[Vgl.][APP.3.1 S. 5]{holgerschildt2022}:
	\begin{compactitem}
		\item Content-Security-Policy
		\item Strict-Transport-Security
		\item Content-Type
		\item X-Content-Type-Options
		\item Cache-Control
	\end{compactitem}
	Diese SOLLTEN zudem immer so restriktiv wie möglich gestaltet werden.

	\mysubparagraph[APP.3.1.A22]{APP.3.1.A22 Penetrationstest und Revision}
	Zur Prüfung auf Sicherheitsprobleme und -verstöße SOLLTEN regelmäßig Penetrationstests
	und Revisionen durchgeführt werden,
	wobei die Ergebnisse protokolliert werden SOLLTEN\footcite[Vgl.][APP.3.1 S. 5]{holgerschildt2022}.

	\myparagraph[APP.3.3.]{Anforderungen bei erhöhtem Schutzbedarf}
	Diese Anforderungen sind dann zu erfüllen,
	wenn der Schutzbedarf über das dem Stand der Technik entsprechende Schutzniveau herausgeht.
	Die Bestimmung dieser erfolgt durch
	eine individuelle Analyse\footcite[Vgl.][APP.3.1 S. 5]{holgerschildt2022}.

	\mysubparagraph[APP.3.1.A20]{APP.3.1.A20 Einsatz von Web Application Firewalls}
	Eine \gls{waf} SOLLTE eingesetzt werden,
	um das Sicherheitslevel zu erhöhen.
	Die Konfiguration dieser ist zudem
	nach jedem Update anzupassen\footcite[Vgl.][APP.3.1 S. 5]{holgerschildt2022}.