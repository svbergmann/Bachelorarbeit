\subsection{Mögliche Frameworks}\label{subsec:moegliche-frameworks}
	Zur Vereinheitlichung der Implementierung sollen wenn möglich \glspl{framework} genutzt werden,
	welche den ganzen Prozess vereinfachen,
	sodass bestenfalls nicht zusätzlich in unterschiedlichen Sprachen programmiert werden muss.
	Da so wenig andere Pakete wie nötig auf dem Zielsystem installiert werden sollen,
	bietet sich die Programmiersprache C\# an.
	Aufgrund von Rahmenbedingungen älterer \hyperref[subsec:betriebssystem]{Betriebssysteme}
	werden im Folgenden die aktuellsten zwei möglichen C\# \glspl{framework} vorgestellt.

	\subsubsection{\netframework}\label{subsubsec:netframework}
		Dieses \gls{framework} ist das in Windows meistgenutzte
		und älteste Konzept,
		um Anwendungen zu erstellen.
		Mit dem Nu-Get Paketmanager lässt sich externer Code einbinden,
		womit unter anderem die Datenbankanbindung leicht und einheitlich implementiert werden kann.
		Hier werden allerdings keine Microservices unterstützt,
		wodurch sich \netframework{} eher zur Programmierung von \glspl{gui} eignet.
		Zudem ist ein kompiliertes \netframework{}-Projekt nicht alleine lauffähig,
		daher wird seit Windows Vista \netframework{} zusammen mit den Windows Updates installiert.

	\subsubsection{\netcore}\label{subsubsec:netcore21}
		\netcore{} wurde am 30.05.2018 von Microsoft zur Nutzung in Visual Studio 2017 Version 15.7 herausgebracht.
		Hiermit wird ein \gls{framework} bereitgestellt,
		welches als Open Source Projekt für Windows,
		MacOS und Linux fungiert.
		Entwickler können damit plattformübergreifende Anwendungen programmieren
		und mit dem NuGet Paketmanager eine Reihe an bereits existierendem Code einbinden und nutzen,
		egal ob dieser von Microsoft oder von anderen Entwicklern in dieser Gemeinschaft stammt.
		So findet sich dort unter anderem Code für die Erstellung von \glspl{webservice}
		und \glslink{authentifizierung}{OAuth 2.0 Authentifizierungsservices}.
		Für die Arbeit mit einer SQL-Datenbank bietet sich vor allem
		das \enquote{Entity Framework Core}-Paket von Microsoft an,
		welches die Anbindung und Nutzung dieser,
		unter anderem mit \gls{linq},
		extrem vereinfacht und standardisiert.

	\subsubsection{Vergleich beider Frameworks}\label{subsubsec:vergleich-beider-frameworks}
		Beide vorangegangenen \glspl{framework} sind also theoretisch möglich und nutzbar.
		Allerdings gibt es einige zu berücksichtigende Gesichtspunkte,
		welche die Entscheidung für ein \gls{framework} vereinfachen.
		Für beide \glspl{webservice} (Daten- und Authentifizierungsservice) wurde nun \nameref{subsubsec:netcore21} genutzt,
		da sich die Erstellung und auch das Hosten hierin einfacher gestaltet.
		In \nameref{subsubsec:netframework} können beispielsweise nur schwierig selbsthostende \glspl{webservice} erstellt werden,
		da Microsoft die \gls{iis} bereitstellt,
		um derartige Dinge zu lösen.
		Will man nun allerdings ein selbsthostendes Programm erstellen,
		wie es in dieser Arbeit der Fall ist,
		so muss dies in \nameref{subsubsec:netcore21} realisiert werden.
		Des Weiteren ist es allerdings so,
		dass in \nameref{subsubsec:netcore21} nur schwierig grafische Anwendungen erstellt werden können.
		Es gestaltet sich also einfacher und fast schon intuitiver,
		auf die Konzepte von \nameref{subsubsec:netframework},
		wie \gls{wpf} oder Windows Forms zurückzugreifen.
		Das Demoprogramm zum Testen der beiden \glspl{webservice} wird also in \nameref{subsubsec:netframework} implementiert.
		Abschließend ist noch zu erwähnen,
		dass Microsoft den \enquote{.NET Standard} als Bindeglied zwischen beiden \glspl{framework} eingeführt hat,
		um die Benutzung gemeinsamer Bibliotheken zu ermöglichen.
		Dies wird auch hier genutzt,
		da die Elemente der Datenbank offensichtlich in beiden Programmen,
		dem \gls{webservice}
		und dem Demo-Programm,
		gleich sind.
		Weiterhin sind in der \vref{tab:netcorevsnetframework} noch einmal einige Unterschiede aufgeführt,
		welche zu der vorhergehenden Entscheidung beigetragen haben.