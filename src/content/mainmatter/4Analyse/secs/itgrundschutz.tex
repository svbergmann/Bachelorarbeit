\section{IT Grundschutz Kompendium}\label{sec:it-grundschutz-kompendium}

	Das IT-Grundschutz-Kompendium ist ein durch den \gls{bsi} herausgegebenes Dokument,
	welches Sicherheitsrisiken im Bereich der Informationssicherheit aufzeigt
	und präventive Maßnahmen dagegen vorstellt.
	Generell definiert dieses Dokument drei Grundwerte
	der Informationssicherheit\footcite[Vlg.][]{holgerschildt2022},
	die es zu schützen gilt:

	\begin{compactitem}
		\item[\textbf{Verlust der Verfügbarkeit:}]
		\label{itm:it-grundschutz-kompendium-verfügbarkeit}
		\mbox{}\\
		Ein Softwaresystem funktioniert nur in Anwesenheit gewisser Daten.
		Wenn diese nicht vorhanden,
		oder nur eingeschränkt verfügbar sind,
		kann dies zur Beeinträchtigung einfacher Aufgaben bis hin zu völliger Stilllegung von Arbeitsprozessen führen.

		\newpage

		\item[\textbf{Verlust der Vertraulichkeit von Informationen:}]
		\label{itm:it-grundschutz-kompendium-vertraulichkeit}
		\mbox{}\\
		Jedes Unternehmen muss mit personenbezogenen Daten von Kunden und Nutzern vertraulich umgehen,
		sodass keine Schäden an der Privatsphäre entstehen.
		Dies sind unter anderem Daten zum Umsatz, Marketing oder Forschung.

		\item[\textbf{Verlust der Integrität (Korrektheit von Informationen):}]
		\label{itm:it-grundschutz-kompendium-korrektheit}
		\mbox{}\\
		Um fehlerhafte Aufträge, o.\ ä.\ vermeiden zu können,
		ist unbedingt sicherzustellen,
		dass alle gespeicherten Daten richtig und unverfälscht sind.
		Die sogenannte \enquote{digitale Identität} spielt zudem eine immer größere Rolle,
		wodurch die Bedrohung falscher Willenserklärungen oder Identitätsfälschungen wachsen.

	\end{compactitem}

	Das Dokument ist zur Eliminierung dieser und anderer Risiken in Bausteine aufgeteilt,
	welche jeweils in:
	\begin{compactenum}
		\item Beschreibung
		\item Gefährdungslage
		\item Anforderungen
		\item Weiterführende Informationen
		\item Anlage: Kreuzreferenztabelle zu elementaren Gefährdungen
	\end{compactenum}
	unterteilt sind.

	\subsection[APP.3.1]{APP.3.1 Webanwendungen und Webservices}\label{subsec:app.3.1-webanwendungen-und-webservices}

		Im Folgenden wird der für diese Arbeit relevante Baustein
		\enquote{APP.3.1 Webanwendungen und Webservices} nach diesem Muster analysiert.

		\subsubsection{Beschreibung}
	\label{subsubsec:beschreibung-app.3.1-webanwendungen-und-webservices}
	Hier wird ein \gls{webservice} in der Einleitung noch einmal abgegrenzt und definiert.
	Ein \gls{webservice} ist anhand dieses Bausteins eine Anwendung,
	welche das \gls{http} oder das \gls{https} verwendet um Ressourcen bereitzustellen.
	In den allermeisten Fällen wird dieser Service nicht von einem Benutzer direkt gesteuert,
	sondern über den Zugriff anderer Anwendungen angesprochen.
	In diesem Baustein geht es zudem ausschließlich um \glspl{webservice}
	mit einer \gls{rest}-Schnittstelle\footcite[Vgl.][APP.3.1 S. 1]{holgerschildt2022}.
		\subsubsection{Gefährdungslage}\label{subsubsec:gefaehrdungslage-app.3.1}
	Die Gefährdungslage dieses Bausteins ist noch einmal unterteilt.
	Im Folgenden werden relevante Gefährdungen angesprochen\footcite[Vgl.][APP.3.2 S. 1]{holgerschildt2022}.

	\myparagraph[APP.3.1.2.1]{Unzureichende Protokollierung von sicherheitsrelevanten Ereignissen}
	Die Protokollierung sicherheitsrelevanter Ereignisse ist erforderlich,
	um später die Ursachen eines bestimmten Ereignisses ermitteln
	und somit Schwachstellen, kritische Fehler,
	oder unerlaubte Änderungen nachvollziehen zu können.

	\myparagraph[APP.3.1.2.2]{Offenlegung sicherheitsrelevanter Informationen bei Webanwendungen und Webservices}
	Bei der Auslieferung von \glspl{webservice} ist darauf zu achten,
	keine sicherheitsrelevanten Daten offenzulegen,
	sprich Informationen über genutzte Programme,
	Systeme oder Versionen weiterzugeben.

	\myparagraph[APP.3.1.2.3]{Missbrauch einer Webanwendung durch automatisierte Nutzung}
	Ein \gls{webservice} kann durch automatisierte Anfragen missbraucht werden
	und somit eventuell Nutzerdaten preisgeben.
	Dadurch könnten Angreifer gültige Benutzernamen sammeln
	und sich dann damit durch mehrfaches Probieren
	von Passwörtern Zugriff zur Anwendung verschaffen.

	\myparagraph[APP.3.1.2.4]{Unzureichende Authentisierung}
	Meistens werden Rollenprofile angelegt,
	um Benutzern die Möglichkeit zu geben,
	bestimmte Ressourcen erreichen zu können,
	welche von anderen Profilen nicht erreicht werden können.
	Falls die \glslink{authentisierung}{Authentisierung} der Nutzer unzureichend ist,
	könnte ein Angreifer einfachen Zugriff erlangen
	und im schlimmsten Fall sogar auf die Nutzerdaten zugreifen,
	falls diese ebenfalls über diese Rolle erreichbar sind.
		\subsubsection{Anforderungen}\label{subsubsec:anforderungen-app.3.1}
	Die Anforderungen an \glspl{webservice} sind wiederum in drei Bereiche unterteilt
	und zudem sollten immer Zuständigkeiten existieren,
	sodass eine Stelle immer \enquote{grundsätzlich zuständig} ist
	und optional auch weitere Zuständigkeiten zugeordnet werden können.

	\myparagraph[APP.3.1]{Basis-Anforderungen}
	Vor allem die Basis Anforderungen MÜSSEN
	laut dem \gls{bsi} erfüllt werden\footcite[Vgl.][APP.3.1 S. 3]{holgerschildt2022}.

	\mysubparagraph[APP.3.1.A1]{APP.3.1.A1 Authentisierung}
	\glspl{webservice} MÜSSEN immer so gestaltet sein,
	dass der Zugriff auf gesperrte Ressourcen nur über eine \gls{authentisierung} erfolgen kann.
	Die Methode hierfür SOLLTE protokolliert werden,
	wobei der Betrieb zudem eine Höchstgrenze
	für fehlgeschlagene Anmeldeversuche festsetzen muss\footcite[Vgl.][APP.3.1 S. 3]{holgerschildt2022}.

	\mysubparagraph[APP.3.1.A4]{APP.3.1.A4 Kontrolliertes Einbinden von Dateien und Inhalten}
	Diese Anforderung entfällt für diesen \gls{webservice},
	da keine Daten hochgeladen werden.

	\mysubparagraph[APP.3.1.A7]{APP.3.1.A7 Schutz vor unerlaubter automatisierter Nutzung}
	Es MUSS garantiert sein,
	dass der \gls{webservice} vor \glslink{autorisierung}{unautorisierter},
	automatisierter Nutzung geschützt ist,
	während das Verhalten der Maßnahmen auf \glslink{autorisierung}{autorisierte} Nutzer
	auch berücksichtigt werden MUSS\footcite[Vgl.][APP.3.1 S. 3]{holgerschildt2022}.

	\mysubparagraph[APP.3.1.A14]{APP.3.1.A14 Schutz vertraulicher Daten}
	Vertrauliche Daten,
	wie beispielsweise Zugangsdaten,
	MÜSSEN serverseitig mit Hilfe von \gls{saltedHashVerfahren} abgesichert sein,
	um diese vor \glslink{autorisierung}{unautorisiertem} Zugriff schützen zu können.

	\myparagraph[APP.3.2]{Standard-Anforderungen}
	Standard-Anforderungen SOLLTEN grundsätzlich zusammen
	mit den \nameref{par:APP.3.1} erfüllt sein\footcite[Vgl.][APP.3.2 S. 4]{holgerschildt2022}.

	\mysubparagraph[APP.3.1.A8]{APP.3.1.A8 Systemarchitektur [Beschaffungsstelle]}
	Bereits während der Planung eines \glspl{webservice} SOLLTEN Sicherheitsrichtlinien
	und -konzepte beachtet werden\footcite[Vgl.][APP.3.1 S. 4]{holgerschildt2022}.

	\mysubparagraph[APP.3.1.A9]{APP.3.1.A9 Beschaffung von Webanwendungen und Webservices}
	Als Ergänzung zu den anderen Anforderungen stellt das IT-Grundschutzkompendium eine Liste bereit,
	welche Eigenschaften bei der Beschaffung eines \glspl{webservice} zusätzlich
	beachtet werden SOLLTEN\footcite[Vgl.][APP.3.1 S. 4]{holgerschildt2022}:
	\begin{compactitem}
		\item sichere Eingabevalidierung und Ausgabekodierung
		\item sicheres Session-Management
		\item sichere kryptografische Verfahren
		\item sichere Authentisierungsverfahren
		\item sichere Verfahren zum serverseitigen Speichern von Zugangsdaten
		\item geeignetes Berechtigungsmanagement
		\item ausreichende Protokollierungsmöglichkeiten
		\item regelmäßige Sicherheitsupdates durch den Entwickler der Software
		\item Schutzmechanismen vor verbreiteten Angriffen auf \webApplications{} und \glspl{webservice}
		\item Zugriff auf den Quelltext der \webApplication{} oder des \glspl{webservice}
	\end{compactitem}

	\mysubparagraph[APP.3.1.A11]{APP.3.1.A11 Sichere Anbindung von Hintergrundsystemen}
	Hintergrundsysteme auf denen Funktionen und Daten ausgelagert sind,
	SOLLTEN einzig über definierte Schnittstellen ansprechbar sein.
	Zudem SOLLTE bei netz- und standortübergreifenden Anwendungen die Kommunikation
	\glslink{authentisierung}{authentisiert}
	und verschlüsselt ablaufen\footcite[Vgl.][APP.3.1 S. 4]{holgerschildt2022}.

	\mysubparagraph[APP.3.1.A12]{APP.3.1.A12 Sichere Konfiguration}
	Der Zugriff bzw.\ die Anfragen auf Ressourcen
	und Methoden eines \glspl{webservice} SOLLTEN derart eingeschränkt sein,
	sodass nur über vorher definierte Pfade kommuniziert werden kann.
	Alle anderen Methoden und nicht benötigten Ressourcen SOLLTEN deaktiviert werden.

	\mysubparagraph[APP.3.1.A21]{APP.3.1.A21 Sichere HTTP-Konfiguration bei Webanwendungen}
	Folgende Response-Header SOLLTEN grundsätzlich
	immer gesetzt sein\footcite[Vgl.][APP.3.1 S. 5]{holgerschildt2022}:
	\begin{compactitem}
		\item Content-Security-Policy
		\item Strict-Transport-Security
		\item Content-Type
		\item X-Content-Type-Options
		\item Cache-Control
	\end{compactitem}
	Diese SOLLTEN zudem immer so restriktiv wie möglich gestaltet werden.

	\mysubparagraph[APP.3.1.A22]{APP.3.1.A22 Penetrationstest und Revision}
	Zur Prüfung auf Sicherheitsprobleme und -verstöße SOLLTEN regelmäßig Penetrationstests
	und Revisionen durchgeführt werden,
	wobei die Ergebnisse protokolliert werden SOLLTEN\footcite[Vgl.][APP.3.1 S. 5]{holgerschildt2022}.

	\myparagraph[APP.3.3.]{Anforderungen bei erhöhtem Schutzbedarf}
	Diese Anforderungen sind dann zu erfüllen,
	wenn der Schutzbedarf über das dem Stand der Technik entsprechende Schutzniveau herausgeht.
	Die Bestimmung dieser erfolgt durch
	eine individuelle Analyse\footcite[Vgl.][APP.3.1 S. 5]{holgerschildt2022}.

	\mysubparagraph[APP.3.1.A20]{APP.3.1.A20 Einsatz von Web Application Firewalls}
	Eine \gls{waf} SOLLTE eingesetzt werden,
	um das Sicherheitslevel zu erhöhen.
	Die Konfiguration dieser ist zudem
	nach jedem Update anzupassen\footcite[Vgl.][APP.3.1 S. 5]{holgerschildt2022}.
		\subsubsection{Weiterführende Informationen}
	\label{subsubsec:weiterfuhrende-informationen-app.3.1-webanwendungen-und-webservices}
	In diesem Kapitel wird auf das \gls{owasp} verwiesen,
	was auch in \vref{sec:owasp} angesprochen und erklärt wird.
	Zudem wird noch das Dokument
	\enquote{Kryptographische Verfahren: Empfehlungen und Schlüssellängen: BSI TR-02102}
	vom \gls{bsi} vorgeschlagen,
	welches Hinweise zur Nutzung kryptographischer Verfahren
	bereitstellt\footcite[Vgl.][APP.3.4 S.6]{holgerschildt2022}.
		\subsubsection{Anlage: Kreuzreferenztabelle zu elementaren Gefährdungen}\label{subsubsec:anlage:app.3.1}

	Hier wird eine Kreuzreferenztabelle (s. \vref{tab:kreuzreferenztabelleApp31}) dargestellt,
	die zeigt welche elementaren
	Gefährdungen\footnote{\cite[Vgl. hierfür][Elementare Gefährdungen S. 1ff]{holgerschildt2022}}
	durch welche \hyperref[subsubsec:anforderungen-app.3.1]{Anforderungen}
	abgeschwächt oder gar eliminiert werden können.
	Folgende elementare Gefährdungen werden behandelt:
	\begin{compactitem}
		\item[\textbf{G 0.14}] Ausspähen von Informationen (Spionage)
		\item[\textbf{G 0.15}] Abhören
		\item[\textbf{G 0.18}] Fehlplanung oder fehlende Anpassung
		\item[\textbf{G 0.19}] Offenlegung schützenswerter Informationen
		\item[\textbf{G 0.21}] Manipulation von Hard- oder Software
		\item[\textbf{G 0.28}] Software-Schwachstellen oder -Fehler
		\item[\textbf{G 0.30}] Unberechtigte Nutzung oder Administration von Geräten und Systemen
		\item[\textbf{G 0.31}] Fehlerhafte Nutzung oder Administration von Geräten und Systemen
		\item[\textbf{G 0.43}] Einspielen von Nachrichten
		\item[\textbf{G 0.46}] Integritätsverlust schützenswerter Informationen
	\end{compactitem}
	Des Weiteren existiert eine Spalte \enquote{CIA},
	welche die Grundwerte der Informationssicherheit,
	wie auf Seite \pageref{itm:it-grundschutz-kompendium-verfügbarkeit} definiert,
	behandelt.
	C (Confidentiality) steht dabei für Vertraulichkeit,
	I (Integrity) für Korrektheit und
	A (Availability) für Verfügbarkeit.



