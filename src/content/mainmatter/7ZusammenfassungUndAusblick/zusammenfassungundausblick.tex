\chapter{Zusammenfassung und Ausblick}\label{ch:zusammenfassung-und-ausblick}

	Vor allem zur Bereitstellung neuer Daten und Formate des \lfidSystems{} sollte ein \gls{webservice} so gestaltet
	und konzipiert werden,
	dass dieser sicher mit den Clienten kommunizieren kann
	und sowohl zuverlässig als auch schnell läuft.
	Hierfür wurde die Grundlage der webbasierten Kommunikation mit dem \nameref{sec:http} geschaffen,
	welches mit \nameref{sec:tls/ssl} chiffriert Daten verschicken kann.
	Allerdings reicht das noch nicht für eine sichere Kommunikation,
	wofür danach \nameref{subsec:oauth-2.0} vorgestellt wurde.
	Dieses Protokoll wird daher zur \gls{autorisierung} der Clienten beim \gls{webservice} genutzt,
	indem die \gls{authentifizierung} auf einen weiteren \gls{webservice} ausgelagert wird.
	Um das alles implementieren zu können wurde danach das Zielsystem analysiert,
	insbesondere welche Sprachen für die Implementierung sinnvoll und realisierbar sind.
	Die Entscheidung fiel auf C\# und \nameref{subsubsec:netcore21} für den \gls{webservice}
	und auf \nameref{subsubsec:netframework} für das Demoprogramm.
	Für die korrekte Gestaltung dieser Anwendung wurde danach das \nameref{sec:it-grundschutz-kompendium} angesprochen,
	wobei aufgrund der schieren Größe dieses Dokuments
	nur auf den Baustein \nameref{subsec:app.3.1-webanwendungen-und-webservices} eingegangen wurde.
	Dieser wurde dann aufgespalten und für jeden Unterpunkt analysiert.
	Wie auch in \nameref{subsubsec:weiterfuhrende-informationen-app.3.1-webanwendungen-und-webservices} beschrieben,
	folgten die Top 10 Bedrohungen des \hyperref[sec:owasp]{\gls{owasp}},
	welche eine etwas allgemeinere Definition der Sicherheitsrisiken im Internet darstellen,
	aber trotzdem gut im Zusammenspiel mit dem \nameref{sec:it-grundschutz-kompendium} funktionieren.
	Dies sieht man auch in der \nameref{tab:itSicherheitsanforderungstabelle},
	da dort jeder Bedrohung eine oder mehrere Anforderungen hinzugefügt worden sind.
	Die Beschreibung des Quellcodes befindet sich im Kapitel \nameref{ch:realisierung},
	wobei unter anderem viel auf die genutzten Attribute des \nameref{subsubsec:netcore21}-\glspl{framework} eingegangen wurde.

	Um an diese Arbeit anzuknüpfen und den \gls{webservice} breiter nutzbar zu machen,
	könnte man
	-- wie auch schon in den \hyperref[sec:optionale-anforderungen]{optionalen Anforderungen} beschrieben --
	auf jeden Fall zu allen Datenbankentitäten Routen erstellen.
	Diese sollten alle \gls{crud}-Operationen widerspiegeln
	und auch derart abgesichert sein,
	dass es einen Unterschied zwischen nur lesenden und bearbeitenden Clienten gibt.
	Weiterhin sollte eine Benutzeroberfläche für den \nameref{subsec:autorisierungsserver} implementiert werden,
	welche den Administrator der Anwendungen dazu befähigt,
	Rollen und Gruppen zu erstellen
	und somit auch Rechte an Clienten zu verteilen.
	Eine Webseite für den \gls{webservice} sollte auch existieren,
	sodass der Administrator gewisse Einstellungen bei dem Programm vornehmen kann.
	Dazu zählen zum Beispiel ein Datenbank Backup speichern und laden,
	neue Daten in die Datenbank einspielen
	und Logging Dateien auslesen können.
	Vor allem aber soll der Verwalter des Systems das Datenbank Passwort setzen können
	und somit die Verbindung zur Datenbank herstellen.
	Zudem muss einer der nächsten Schritte sein,
	die Programme auf die neuste Version von .NET (aktuell .NET 6.0) umzustellen
	um dadurch Sicherheitslücken und bekannte Fehler beheben zu können.