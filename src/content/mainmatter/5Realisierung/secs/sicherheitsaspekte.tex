\section{Sicherheitsaspekte}\label{sec:sicherheitsaspekte}

	Für die bessere Übersicht der realisierten und behandelten Sicherheitsaspekte wird alles,
	was im Folgenden behandelt wird,
	noch einmal in \vref{tab:itSicherheitsanforderungstabelle} kurz zusammengefasst.
	Zur Beibehaltung der Struktur werden die Risiken der \nameref{sec:owasp}
	zu einigen Anforderungen des \hyperref[sec:it-grundschutz-kompendium]{IT Grundschutz Kompendiums}
	zugeordnet und -- wenn möglich -- chronologisch abgearbeitet.

	Die Bedrohung \nameref{par:A01:2021} wird dadurch abgeschwächt,
	dass es bei diesem \gls{webservice} zwar keine klassischen Nutzer gibt,
	jedoch jedes \glslink{autorisierung}{autorisierte} Programm eigene Rechte
	und Bereiche zugewiesen bekommt,
	welche freigegeben und erreicht werden dürfen.
	Damit wird vor allem die \hyperref[par:APP.3.1]{Basis Anforderung}
	\nameref{subpar:APP.3.1.A1} neben anderen erfüllt.

	Durch die Nutzung von \nameref{sec:tls/ssl} wird \nameref{par:A02:2021} entgegengewirkt,
	da somit die übertragenen Daten generell verschlüsselt sind
	und auch gar nicht erst verschickt werden,
	falls der Kommunikationspartner nicht \glslink{autorisierung}{autorisiert} ist.
	Dies kann den \hyperref[par:APP.3.2]{Standard-Anforderungen}
	\nameref{subpar:APP.3.1.A8}, \nameref{subpar:APP.3.1.A9}
	und \nameref{subpar:APP.3.1.A11}
	zugeordnet werden.

	Da \nameref{subsubsec:netcore21} mit einem \gls{framework} zur Erstellung eines \glspl{webservice} genutzt wird,
	wird bei jeder Route automatisch gefragt,
	ob der gegebene Parameter dem Übergabewert entspricht.
	Zudem sind die Implementierungen der Methoden mit \gls{linq} gestaltet,
	was bedingt,
	dass intern Microsofts Entity Framework für die Datenbankabfragen genutzt wird,
	wodurch die übergebenen Werte noch zusätzlich geprüft werden
	und quasi keine \gls{sql}-Injektion mehr möglich ist.
	Hiermit wird \nameref{subpar:APP.3.1.A14},
	\nameref{subpar:APP.3.1.A9}
	und \nameref{subpar:APP.3.1.A11} erfüllt
	sowie \nameref{par:A03:2021} entgegen gewirkt.

	Die vierte Bedrohung, \nameref{par:A04:2021},
	bezieht sich vor allem auf \nameref{subpar:APP.3.1.A8} und \nameref{subpar:APP.3.1.A9},
	die Beschaffung von externem Material,
	da vor allem auch Fremdcode auf Sicherheit geprüft werden muss.
	Dieses Programm benutzt daher auch nur Pakete,
	welche entweder direkt von Microsoft zur Verfügung gestellt werden oder sehr viele Nutzer
	und positive Einträge in Bezug auf die Sicherheit hat.
	Zudem wird bei jedem Paket oder auch \gls{framework} darauf geachtet,
	dass der aktuellste Patch der jeweiligen Version genutzt wird,
	keine Versionskonflikte entstehen
	und notwendige Updates durchgeführt werden.
	Damit wird auch direkt
	\nameref{par:A06:2021}
	und \nameref{par:A08:2021} entgegen gewirkt.
	Leider kann dies aber hinsichtlich der Rahmenbedingungen des \hyperref[subsec:betriebssystem]{Betriebssystems}
	nicht bei .NET Core gewährleistet werden.

	Da der \gls{webservice} so offen wie nötig,
	aber so restriktiv wie möglich gestaltet werden soll und wurde,
	wird  \nameref{par:A05:2021}
	und \nameref{par:A07:2021} angegangen.
	Speziell können \nameref{subpar:APP.3.1.A1},
	\nameref{subpar:APP.3.1.A7},
	\nameref{subpar:APP.3.1.A14},
	\nameref{subpar:APP.3.1.A9}
	und \nameref{subpar:APP.3.1.A12} dieser Bedrohung zugeordnet werden.
	Von Anfang an wurden nur \glslink{autorisierung}{autorisierten} Clienten Methoden freigegeben
	und keine Standardpasswörter oder -benutzerkennungen benutzt,
	wobei eine zusätzliche Sicherheitsebene durch den implementierten \nameref{subsec:oauth-2.0}-Standard,
	sowie dem Auslaufen der somit generierten \hyperref[par:access-token]{Access Tokens} geschaffen wurde.

	Zur Aufzeichnung der Aktionen auf dem \gls{webservice} sollten während jeder Sitzung Logging-Dateien erstellt werden,
	was auch von \nameref{subpar:APP.3.1.A1} gefordert ist.
	Um beispielsweise fehlgeschlagene Anmeldeversuche erkennen
	und nach Gefährlichkeit einstufen zu können,
	müssen diese Informationen sicher gespeichert werden
	und insbesondere dürfen diese nicht an den Nutzer weitergeben werden.
	Damit wird \nameref{par:A09:2021} vereitelt,
	falls unter anderem bei einem fehlgeschlagenen \glslink{authentifizierung}{Authentifizierungsversuch}
	dieser gespeichert wird
	und an den Nutzer nur die Meldung zurückgegeben wird,
	dass der Versuch nicht erfolgreich war.
	Zusätzlich wird durch die Protokollierung \nameref{subpar:APP.3.1.A7} erfüllt.
	Hier werden alle Aktionen des \glspl{webservice} in einer Datei gespeichert,
	welche von außen nicht einsehbar ist,
	um alles nachvollziehen zu können.

	Die letzte zu behandelnde Bedrohung \nameref{par:A10:2021}
	zielt eher auf eine Gefährdung für den Nutzer
	und nicht für den bereitstellenden Dienst ab.
	Trotzdem kann damit noch einmal die Anforderung \nameref{subpar:APP.3.1.A12} angesprochen
	und realisiert werden,
	da schon mit einer restriktiven Freigabe von Methoden sehr viel erreicht wird,
	was sich auch in der Implementierung wiederfindet.

