\subsection{Tests}\label{subsec:tests}
	Die hier aufgeführten Tests existieren vor allem,
	um die Unterschiede der Request-Zeiten,
	wie in \vref{fig:timeOfRequests} dokumentiert,
	zu zeigen.
	Die Klasse \inlinelst{Authorized\-Client\-Fixture} zeigt dabei,
	wie sich ein Client mit dem \gls{webservice} \glslink{autorisierung}{autorisieren} und verbinden kann.
	Zeile 7--11 in \vref{lst:authorizedClientFixture} setzt die nötigen Informationen
	für den \glslink{authentifizierung}{Authentifizierungsprozess},
	wie die \gls{url} des \hyperref[sec:authorization-server]{Authorisierungsservers},
	die \inlinelst{clientId},
	den \inlinelst{scope} und das \inlinelst{clientSecret}.
	Diese Parameter werden dann für die Funktion \inlinelst{Request\-Token\-To\-Authorization\-Server} genutzt,
	welche das \hyperref[par:access-token]{Access Token} beantragt
	und bei Erfolg zurückgibt.
	Für die Beantragung werden \gls{http}-Header Felder genutzt.
	Nach der Rückkehr aus dieser Methode wird das \hyperref[par:access-token]{Access Token} aus dem \gls{json}-Format geparst
	und an den Clienten weitergegeben,
	welcher nun ein Header-Feld namens \enquote{Bearer} mit dem \hyperref[par:access-token]{Access Token} als Wert setzen kann.
	Dieser Client wird dann genutzt,
	um \glslink{autorisierung}{autorisierte} Anfragen an den \nameref{sec:webservice} zu schicken.
	In dem in \vref{lst:countriesWebTestAuthorized} gezeigten Fall werden alle vorhandenen Länder angefragt.
	Der Request an sich findet in der Methode \inlinelst{GetCountryModels} in Zeile 28--36 statt,
	wobei mit einem asynchronen Aufruf der gegebenen \gls{uri} die Ressource angefragt wird.
	Diese Liste der Länder wird dann deserialisiert
	und als \inlinelst{List} Objekt zurückgegeben.
	Die gleiche Methode findet sich in \vref{lst:countriesWebTestUnauthorized} wieder,
	mit der einzigen Unterscheidung,
	dass dieser Client nicht \glslink{autorisierung}{autorisiert} ist.
	Der letzte Test (s. \vref{lst:countriesSQLTest}) zeigt,
	wie sich ein Programm direkt mit der Datenbank verbinden kann
	und Daten über eine \gls{sql}-Anfrage erhält.