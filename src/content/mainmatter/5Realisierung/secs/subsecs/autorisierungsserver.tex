\subsection{Autorisierungsserver}\label{subsec:autorisierungsserver}

	Der Autorisierungsserver stellt das Herzstück der
	implementierten \nameref{subsec:oauth-2.0} \gls{autorisierung} dar.
	Wie in \vref{lst:authorizationServerProgram} zu sehen,
	wird hier ein neuer \gls{webservice} erstellt,
	welcher auf den Port 5003 (s. Zeile 15) hört und das \gls{https} nutzt.
	In Zeile 14 wird festgelegt,
	dass die Klasse \inlinelst{Startup} genutzt werden soll,
	welche in \vref{lst:authorizationServerStartup} dargestellt ist.
	Diese wird dann per \gls{constructorInjection} erstellt
	und bekommt ein Objekt vom Typen \inlinelst{IConfiguration}.
	Die eigentlich wichtige Methode für den Autorisierungsserver ist \inlinelst{ConfigureServices}.
	Diese bekommt eine Kollektion vom Typen \inlinelst{IServiceCollection} übergeben
	und fügt dieser über einige Methodenaufrufe Funktionen und Eigenschaften hinzu,
	sodass eine \gls{autorisierung} stattfinden kann.
	In Zeile 12 und 13 werden die vordefinierten Bereiche
	und die daraus resultierenden Clienten
	über die Methoden \inlinelst{Config.GetApiResources}
	und \inlinelst{Config.GetClients} hinzugefügt.
	In \vref{lst:authorizationServerConfig} sind eben diese Methoden vorhanden.
	Zeile 5 und 6 zeigen, dass zwei \enquote{scopes} erstellt werden,
	einer nur mit Leserechten, der andere mit vollen Rechten.
	Diese werden dann wiederum in der Methode \inlinelst{GetClients} genutzt,
	in welcher zwei Clienten erstellt werden.
	Der erste Client (Zeile 12--20) kann hier wieder nur lesen
	und bekommt daher auch genau diesen \enquote{scope} zugewiesen,
	während der andere (Zeile 21--29) wieder alles darf.
	Des Weiteren existiert in \inlinelst{Startup} noch die Methode \inlinelst{Configure},
	welche vor allem dafür zuständig ist,
	dieses Programm als Identitätsserver darzustellen
	und \gls{https} zu nutzen.
	Zusätzlich wird noch entschieden,
	ob entwicklerspezifische Fehlercodes angezeigt werden sollen oder nicht,
	was davon abhängt,
	ob der Aufruf des Programms im \enquote{Debug}
	oder \enquote{Release} Modus stattfindet.