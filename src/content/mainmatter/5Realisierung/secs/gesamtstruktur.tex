\section{Gesamtstruktur}\label{sec:gesamtstruktur}

	Das entwickelte Softwaresystem besteht mittlerweile aus drei unterschiedlichen Komponenten,
	welche für einen reibungslosen Betrieb gut zusammenarbeiten müssen.
	Diese Programme werden dafür benötigt,
	die \gls{authentifizierung} durchzuführen,
	Daten abzufragen und diese letztendlich auch auszuwerten.
	Im Folgenden wird der Code von den einzelnen Bestandteile analysiert und erklärt,
	wobei nur auf relevante Zeilen/Methoden/Aufrufe eingegangen wird.
	Die selbsterklärenden Codefragmente bleiben größtenteils unerwähnt
	um die wichtigen Zeilen/Aufrufe hervorheben zu können,
	sodass der Code aber trotzdem genau so lauffähig wäre.
	Zusätzlich ist noch zu erwähnen,
	dass \nameref{subsubsec:netcore21} per Konstruktion zwei Klassen benötigt
	um einen \gls{webservice} erstellen zu können.
	Die Klasse \lstinline!Program.cs! beinhaltet hierbei die \lstinline!Main! Methode,
	welche intern die Klasse \lstinline!Startup.cs! erstellt.

	\subsection{Autorisierungsserver}\label{subsec:autorisierungsserver}

	Der Autorisierungsserver stellt das Herzstück der
	implementierten \nameref{subsec:oauth-2.0} \gls{autorisierung} dar.
	Wie in \vref{lst:authorizationServerProgram} zu sehen,
	wird hier ein neuer \gls{webservice} erstellt,
	welcher auf den Port 5003 (s. Zeile 15) hört und das \gls{https} nutzt.
	In Zeile 14 wird festgelegt,
	dass die Klasse \inlinelst{Startup} genutzt werden soll,
	welche in \vref{lst:authorizationServerStartup} dargestellt ist.
	Diese wird dann per \gls{constructorInjection} erstellt
	und bekommt ein Objekt vom Typen \inlinelst{IConfiguration}.
	Die eigentlich wichtige Methode für den Autorisierungsserver ist \inlinelst{ConfigureServices}.
	Diese bekommt eine Kollektion vom Typen \inlinelst{IServiceCollection} übergeben
	und fügt dieser über einige Methodenaufrufe Funktionen und Eigenschaften hinzu,
	sodass eine \gls{autorisierung} stattfinden kann.
	In Zeile 12 und 13 werden die vordefinierten Bereiche
	und die daraus resultierenden Clienten
	über die Methoden \inlinelst{Config.GetApiResources}
	und \inlinelst{Config.GetClients} hinzugefügt.
	In \vref{lst:authorizationServerConfig} sind eben diese Methoden vorhanden.
	Zeile 5 und 6 zeigen, dass zwei \enquote{scopes} erstellt werden,
	einer nur mit Leserechten, der andere mit vollen Rechten.
	Diese werden dann wiederum in der Methode \inlinelst{GetClients} genutzt,
	in welcher zwei Clienten erstellt werden.
	Der erste Client (Zeile 12--20) kann hier wieder nur lesen
	und bekommt daher auch genau diesen \enquote{scope} zugewiesen,
	während der andere (Zeile 21--29) wieder alles darf.
	Des Weiteren existiert in \inlinelst{Startup} noch die Methode \inlinelst{Configure},
	welche vor allem dafür zuständig ist,
	dieses Programm als Identitätsserver darzustellen
	und \gls{https} zu nutzen.
	Zusätzlich wird noch entschieden,
	ob entwicklerspezifische Fehlercodes angezeigt werden sollen oder nicht,
	was davon abhängt,
	ob der Aufruf des Programms im \enquote{Debug}
	oder \enquote{Release} Modus stattfindet.
	\subsection{Webservice}\label{subsec:webservice}

	Das eigentliche Programm nutzt in \vref{lst:webServiceProgram} fast
	den gleichen Code wie der \nameref{subsec:autorisierungsserver},
	da beide Programme einen \gls{webservice} darstellen.
	Einzig der Port auf den dieser \gls{webservice} hört unterscheidet sich (s. Zeile 15),
	um beide gleichzeitig nutzen zu können.
	Ansonsten wird hier ebenfalls \gls{constructorInjection} genutzt
	und dadurch die Klasse \inlinelst{Startup} erstellt,
	welche sich in den Methoden \inlinelst{Configure\-Services} und \inlinelst{Configure} unterscheidet.
	Zur besseren Übersichtlichkeit wird daher \inlinelst{Configure\-Services}
	in \vref{lst:webServiceConfigureServices}
	und \inlinelst{Configure} in \vref{lst:webServiceConfigure} dargestellt.
	Beide Methoden konfigurieren den zu erstellenden \gls{webservice}
	und werden intern beim Programmstart aufgerufen.

	\subsubsection{ConfigureServices}\label{subsubsec:configureservices}
		In Zeile 7 wird ein \gls{singleton} einer \inlinelst{Auto\-Mapper} Klasse angelegt
		und den Services hinzugefügt,
		welches später dafür sorgen wird,
		dass die Objekte aus der Datenbank zu sogenannten \enquote{shallow Objects} zugeordnet werden.
		Das ist sinnvoll,
		da nicht alle Attribute aus der Datenbank zwingend in diesen Objekten präsentiert werden müssen.
		Zeile 13 fügt ebenfalls \glspl{singleton} hinzu,
		die sich um die \gls{autorisierung} und die \gls{api}-Verwaltung,
		sprich, wo welche Methoden stehen, kümmern.
		Die Zeilen 15--34 bearbeiten weiter die \gls{autorisierung} und
		verlinken unter anderem die Zugangsberechtigungen
		und den \nameref{subsec:autorisierungsserver} mit diesem \gls{webservice}.
		Dafür werden noch Attribute definiert,
		die als Optionen zu den \httpMethoden{} zugeordnet werden können,
		um die Berechtigungen einzuschränken.
		Für die einfachere Entwicklung und grafische Ansicht der nutzbaren \httpMethoden{} wird
		Swagger\footcite[S.][]{smartbear_2022} genutzt und in den folgenden Zeilen eingerichtet.
		Ein weiterer Vorteil hiervon ist,
		dass durch Swagger automatisch eine XML-Datei erstellt wird,
		welche in der \gls{openApiSpec} 3.0\footcite[S.][]{miller_whitlock_gardiner_ralphson_ratovsky_sarid_2017} formatiert ist.
		Zuletzt wird noch die \gls{sql}-Datenbank über einen \inlinelst{DbContext} hinzugefügt und angesprochen,
		indem die Verbindungszeichenfolge aus einer Konfigurationsdatei ausgelesen wird.

	\subsubsection{Configure}\label{subsubsec:configure}
		Hier wird neben anderen Konfigurationen ein selbst geschriebener ExceptionHandler der Applikation hinzugefügt,
		welcher angesprochen wird,
		falls bei der Ausführung irgendeiner \httpMethode{} eine Ausnahme geworfen wird.
		In dem internen \inlinelst{switch-\-case} Konstrukt wird zwischen einer
		\inlinelst{Argument\-Out\-Of\-Range\-Exception},
		einer \inlinelst{Data\-Not\-Found\-Exception}\footnote{S. \vref{lst:webServiceDataNotFoundException}}
		und allen anderen Exceptions unterschieden.
		Es werden jeweils die Fehlermeldungen in \gls{json} umgewandelt
		und zusammen mit einem entsprechenden \httpStatusCode{} zurückgegeben.
		Die letzten Zeilen setzen dann nur noch einige Attribute des Programms.

	\subsubsection{ApiControllerBase}\label{subsubsec:apicontrollerbase}
		Dies ist die Elternklasse,
		welche von allen anderen Controller-Klassen implementiert wird
		und einige nützliche Attribute bereitstellt.
		Die Annotationen über dem Klassennamen von Zeile 2--6 werden ebenfalls weitervererbt.
		Über \inlinelst{[ApiController]} wird die Klasse als Controller gekennzeichnet
		und kann nun \httpMethoden{} implementieren,
		welche alle über \inlinelst{[Produces(\enquote{application/json})]} eine in \gls{json} formatierte Datei zurückgeben.
		Um diese Methoden danach aufrufen zu können,
		wird mit \inlinelst{[Route(\enquote{api/controller})]} festgelegt,
		dass die Adresse mit \inlinelst{api/<Name des Controllers>/} beginnt.
		Die \gls{autorisierung} wird in den Zeilen 5--6 gesteuert,
		indem Schemata und Policy gesetzt werden,
		welche allerdings noch für einzelne Methoden überschrieben werden können.
		Diese Klasse arbeitet ebenfalls mit \gls{constructorInjection} und bekommt daher einige Objekte,
		die dafür genutzt werden,
		die Klassenattribute zu setzen um damit weiterarbeiten zu können.
		Der übergebene \inlinelst{IAction\-Descriptor\-Collection\-Provider} enthält eine
		zuvor generierte Liste aller verfügbaren Methoden und Links.
		Wie schon zuvor angesprochen,
		ist es sinnvoll,
		nicht alle Attribute des Datenbankobjektes preiszugeben,
		wofür das Objekt des Typen \inlinelst{IMapper} zuständig ist.
		Des Weiteren kann jeder Controller mit dem \inlinelst{lfid\-Context}
		Daten aus der Datenbank abfragen und der \inlinelst{logger}
		ist natürlich zur Protokollierung da.

	\subsubsection{Controller}\label{subsubsec:controller}
		In \vref{lst:countriesController} ist beispielhaft ein Controller dargestellt,
		welcher \nameref{subsubsec:apicontrollerbase} implementiert.
		Es wird hier dargestellt,
		wie eine \hyperref[itm:httpget]{GET}
		und eine \hyperref[itm:httphead]{HEAD} Route
		erstellt werden kann und wie die Datenbankabfrage funktioniert.

		Im Konstruktor passiert nichts,
		außer dass der Super-Konstruktor wieder durch \gls{constructorInjection} aufgerufen wird.
		Die eigentliche Arbeit passiert in der Methode \inlinelst{Get\-Shadow\-Countries},
		wobei diese durch die Annotationen in Zeile 7 und 8 eine \hyperref[itm:httpget]{GET}
		und eine \hyperref[itm:httphead]{HEAD} Methode darstellt.
		Intern wird hier über das \inlinelst{LfidContext} Objekt eine Menge aller Länder abgefragt.
		Falls das resultierende Objekt leer ist,
		wird eine \inlinelst{Data\-Not\-Found\-Exception}\footnote{S. \vref{lst:webServiceDataNotFoundException}} geworfen
		und ansonsten wird damit weitergearbeitet.
		Die Zeilen 12--15 stellen eine \gls{linq} Abfrage dar,
		welche die zurückgegebene Menge zuerst nach
		\enquote{CustareaCode}\footnote{Der CustareaCode stellt einen im \gls{arinc}-Format spezifizierten Code für die korrespondierende Region dar.}
		und dann nach
		\enquote{IcaoCode}\footnote{Der \gls{icao}-Code dient zur eindeutigen Identifizierung von Flugplätzen. Da die ersten zwei Buchstaben das Land beschreiben, werden diese hier auch in der Tabelle der Länder geführt.}
		sortiert.
		Die Elemente werden dann über den \enquote{IcaoCode} gruppiert
		und schließlich in ein \inlinelst{Country\-Model} gemappt.
		Mit dem Aufruf von \inlinelst{return Ok(...)} wird die resultierende Menge
		als formatiertes \gls{json}-Objekt zurückgegeben.
		Das Rückgabeobjekt ist eine \inlinelst{Task} von \inlinelst{Action\-Result} von \inlinelst{IEnumerable}
		von \inlinelst{Country\-Model},
		um unter anderem die Methode asynchron gestalten zu können.

	\subsection{Tests}\label{subsec:tests}
	Die hier aufgeführten Tests existieren vor allem,
	um die Unterschiede der Request-Zeiten,
	wie in \vref{fig:timeOfRequests} dokumentiert,
	zu zeigen.
	Die Klasse \inlinelst{Authorized\-Client\-Fixture} zeigt dabei,
	wie sich ein Client mit dem \gls{webservice} \glslink{autorisierung}{autorisieren} und verbinden kann.
	Zeile 7--11 in \vref{lst:authorizedClientFixture} setzt die nötigen Informationen
	für den \glslink{authentifizierung}{Authentifizierungsprozess},
	wie die \gls{url} des \hyperref[sec:authorization-server]{Authorisierungsservers},
	die \inlinelst{clientId},
	den \inlinelst{scope} und das \inlinelst{clientSecret}.
	Diese Parameter werden dann für die Funktion \inlinelst{Request\-Token\-To\-Authorization\-Server} genutzt,
	welche das \hyperref[par:access-token]{Access Token} beantragt
	und bei Erfolg zurückgibt.
	Für die Beantragung werden \gls{http}-Header Felder genutzt.
	Nach der Rückkehr aus dieser Methode wird das \hyperref[par:access-token]{Access Token} aus dem \gls{json}-Format geparst
	und an den Clienten weitergegeben,
	welcher nun ein Header-Feld namens \enquote{Bearer} mit dem \hyperref[par:access-token]{Access Token} als Wert setzen kann.
	Dieser Client wird dann genutzt,
	um \glslink{autorisierung}{autorisierte} Anfragen an den \nameref{sec:webservice} zu schicken.
	In dem in \vref{lst:countriesWebTestAuthorized} gezeigten Fall werden alle vorhandenen Länder angefragt.
	Der Request an sich findet in der Methode \inlinelst{GetCountryModels} in Zeile 28--36 statt,
	wobei mit einem asynchronen Aufruf der gegebenen \gls{uri} die Ressource angefragt wird.
	Diese Liste der Länder wird dann deserialisiert
	und als \inlinelst{List} Objekt zurückgegeben.
	Die gleiche Methode findet sich in \vref{lst:countriesWebTestUnauthorized} wieder,
	mit der einzigen Unterscheidung,
	dass dieser Client nicht \glslink{autorisierung}{autorisiert} ist.
	Der letzte Test (s. \vref{lst:countriesSQLTest}) zeigt,
	wie sich ein Programm direkt mit der Datenbank verbinden kann
	und Daten über eine \gls{sql}-Anfrage erhält.
	\subsection{Demo Anwendung}\label{subsec:demo-anwendung}
	Diese Anwendung wird zur Nachbildung der Simulatorkomponente genutzt,
	welche später die Daten abfragen soll.
	Der Code für die GUI ist hier extra nicht erwähnt,
	da dieser nicht wirklich relevant für die Arbeit ist
	und die Erklärung unnötig verkompliziert.
	Die Klasse \inlinelst{MainWindow}\footnote{S. \vref{lst:demoApplicationApplication}}
	stellt im Kern auch nur einen \glslink{autorisierung}{autorisierten} Clienten bereit,
	wie auch schon in \vref{lst:authorizedClientFixture} erklärt.
	Wie und über welche Routen die Requests abgeschickt werden,
	wird in \vref{lst:demoApplicationFetchData} dargestellt.
	Diese sind allerdings fast die gleichen Methoden,
	die auch in den \nameref{subsec:tests} genutzt werden.


