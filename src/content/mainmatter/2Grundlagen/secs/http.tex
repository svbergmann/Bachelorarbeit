\section{HTTP}\label{sec:http}
	1996 wurde die erste Version des \gls{http}
	-- \gls{http}/1.0\footcite[Vgl.][]{rfc1945} --
	veröffentlicht.
	Seither wird das Protokoll vor allem für kleinere \glspl{webservice} genutzt,
	da es schnell, leichtgewichtig und einfach zu implementieren ist.
	Um die Ladezeiten des Protokolls zu verbessern,
	wurde 1999 \enquote{\gls{http}/1.1}\footcite[Vgl.][]{rfc7230} veröffentlicht und standardisiert.
	Es existierte nun ein Header Feld namens \enquote{Connection: keep-alive},
	welches vom Clienten dazu genutzt werden kann,
	die darunterliegende \glslink{tcp}{TCP-Verbindung} aufrechtzuerhalten.
	Das \enquote{\gls{http}/2.0}\footcite[Vgl.][]{rfc7540} stellte eine erneute Beschleunigung der Übertragung dar.
	Mit einer Abwärtskompatibilität zur Version 1.1 wurde diese Version 2015 standardisiert
	und kann nach einer Anfrage des Clienten mit dem Header Feld \enquote{Connection: Upgrade} genutzt werden.
	Die Neuheit war nun das \enquote{Multiplexing},
	was eine Aufteilung der Datenübertragung in Streams darstellt.
	Diese Streams können nun mehrere Anfragen oder Antworten parallel übertragen,
	wobei nicht auf die Reihenfolge geachtet werden muss,
	da jeder Stream eine eindeutige ID besitzt.

	Für jede Aktion werden unterschiedliche Methoden im \gls{http} genutzt.
	So existiert unter anderem zu jeder Operation aus dem Akronym \gls{crud} eine äquivalente \httpMethode.
	Die Methoden werden im Folgenden aufgeführt und erklärt und
	-- falls möglich --
	einer \crudOperation{} zugewiesen.
	\begin{compactitem}
		\item{\textbf{GET}}\label{itm:httpget} fordert eine im Request-\glsname{uri}
		definierte Ressource an. (\textbf{READ})

		\item{\textbf{HEAD}}\label{itm:httphead} ist ähnlich wie GET, übermittelt aber nur den Header ohne den Body.

		\item{\textbf{POST}}\label{itm:httppost} überträgt Daten an den Server. (\textbf{CREATE})

		\item{\textbf{PUT}}\label{itm:httpput} möchte alle Daten, welche momentan auf dem Server liegen,
		mit den eigenen Daten überschreiben. (\textbf{UPDATE})

		\item{\textbf{DELETE}}\label{itm:httpdelete} löscht Dokumente auf dem Server
		(sofern die Rechte dazu vorhanden sind). (\textbf{DELETE})

		\item{\textbf{CONNECT}}\label{itm:httpconnect} kann einen Tunnel zum Server
		für die bidirektionale Kommunikation öffnen.

		\item{\textbf{OPTIONS}}\label{itm:httpoptions} fordert den Zugriffspfad einer Ressource an
		und die weitere Kommunikation, welche darüber möglich ist.

		\item{\textbf{TRACE}}\label{itm:httptrace} ist zum Testen.
		Die Anfrage wird vom Server zurückgeschickt.
	\end{compactitem}