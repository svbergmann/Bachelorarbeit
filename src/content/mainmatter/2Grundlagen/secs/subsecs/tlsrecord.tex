\subsection{TLS Record Protocol}\label{subsec:tls-record-protocol}
	Das Record Protokoll wird für die eigentliche Datenübertragung genutzt
	und ist vollständig abgekoppelt vom \nameref{subsec:tls-handshake-protocol}.
	Zuerst werden die zu verschickenden Daten in handliche Blöcke aufgeteilt,
	welche optional auch komprimiert werden können.
	Jeder Nachricht wird dann ein Nachrichtenauthentifizierungscode (engl.\ \gls{mac}) zugewiesen,
	der zur Integritätsprüfung genutzt wird.
	Final werden noch genau diese Daten
	mit dem im \nameref{subsec:tls-handshake-protocol} ausgehandelten Schlüssel
	chiffriert.
	Sind die Daten nun vollständig verschlüsselt und verschickt,
	wird die empfangende Seite das Protokoll noch einmal rückwärts durchlaufen,
	um die Daten wieder darstellen zu können.