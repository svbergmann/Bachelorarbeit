\subsection{API-Keys}\label{subsec:api-keys}
	Die \gls{api}-Keys eignen sich zur einfachen \gls{autorisierung} eines Clienten
	und werden häufig von Programmen genutzt,
	bei welchen kein direkter Zugriff auf sensible Daten nötig ist.
	Hierzu wird bei beiden Partnern ein gemeinsamer Schlüssel hinterlegt,
	wodurch der Kommunikationspartner identifiziert werden kann.
	Dieser Schlüssel sollte einmalig vergeben werden und wird im Falle von \gls{rest}
	und \gls{http} entweder über die Request-Query bei einem POST-Request
	oder als Header oder \gls{cookie} bei einem GET-Request übergeben.
	Der \gls{webservice} prüft basierend auf dem Schlüssel die genehmigten oder gesperrten Ressourcen
	und wertet anhand der \gls{autorisierung} die Anfrage aus.
	Dieses Verfahren ist stark durch \glspl{manInTheMiddleAngriff} bedroht,
	was unter anderem dazu führt,
	dass es nur unter Benutzung des \gls{https} als sicher gilt,
	da ansonsten der Schlüssel abgefangen und genutzt werden kann.