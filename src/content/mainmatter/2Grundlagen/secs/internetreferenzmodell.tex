\section{Internetreferenzmodell}\label{sec:internetreferenzmodell}
	Um die nächsten Kapitel in den Gesamtzusammenhang einordnen zu können,
	wird folgend kurz auf die \enquote{Grundlage des Internets} eingegangen.

	Im Allgemeinen existieren zwei Modelle:
	Einmal das \isoOsi,
	welches ab 1977 durch die \gls{iso}
	für die Kommunikation in offenen Netzwerken (\gls{osi})
	bereitgestellt wurde
	und eine (etwas einfachere und) neuere Version,
	das \tcpIp,
	bei dem der Name vom \gls{tcp} und vom \gls{ip} stammt.
	Diese sogenannten Schichtenmodelle definieren einen Stapel von Ebenen,
	welche versuchen,
	die Aufgaben zu abstrahieren,
	voneinander abzukoppeln und damit zu vereinfachen.
	In \vref{fig:schichtenmodell} sind die verschiedenen Lagen der Modelle gegenübergestellt
	und Beispiele genannt,
	welche Protokolle in welcher Schicht liegen.

	Fängt man nun an,
	die Schichten von unten nach oben durchzuzählen,
	befindet sich auf der ersten Schicht im \isoOsi{} die Bitübertragungsschicht,
	welche zusammen mit der zweiten Schicht,
	der Sicherungsschicht,
	die \enquote{Host-to-Network} Schicht im \tcpIp{} ergibt.
	Diese Ebene ist dafür verantwortlich,
	die physikalische Übertragung zu gewährleisten.
	Beispiele hierfür sind vor allem das Ethernet (Kabel)
	und das WLAN nach \glslink{ieee}{IEEE} 802.11 Standard,
	wobei einfache Fehlererkennungsverfahren zusätzlich darin angesiedelt sind.

	Die nächste Schicht beinhaltet das \gls{ip}
	und ist damit zuständig für die Adressierung und Übertragung der Pakete.
	Die Reihenfolge der Pakete muss dabei allerdings nicht zwingend eingehalten werden.
	Zudem ist es möglich,
	dass Übertragungsfehler entstehen,
	welche jedoch durch diese Schicht nicht erkannt werden können.
	Die Internet-, bzw.\ Vermittlungsschicht stellt auch noch das \gls{icmp} bereit,
	welches den Versand von Kontrollinformationen
	-- beispielsweise Änderung von Routing, oder Statusabfragen --
	möglich macht.

	In beiden Modellen folgt darauf die \enquote{Transportschicht}.
	Diese ist für die Kommunikation zwischen zwei Partnern zuständig und beinhaltet zwei Protokolle.
	Das \gls{tcp} ist zuverlässig und verbindungsorientiert,
	weshalb es für kritische und sicherheitsrelevante Übertragungen genutzt wird.
	Das \gls{udp} ist verbindungslos und unzuverlässig,
	jedoch dadurch schneller,
	wodurch es zur Übertragung großer Datenströme --
	wie beispielsweise Videostreams -- genutzt werden kann,
	wobei das Fehlen einzelner kleinerer Pakete nicht auffällt.

	Die drei ursprünglichen Schichten (Sitzung, Darstellung und Anwendung) im \isoOsi{}
	wurden im \tcpIp{} zu einer Schicht,
	der Verarbeitungsschicht,
	zusammengefasst.
	Dieser Bereich dient zur Bereitstellung höherwertiger Protokolle und Anwendungssoftware.
	Hierzu gehören unter anderem das \gls{ftp}, das \gls{smtp} und auch das \gls{http}.